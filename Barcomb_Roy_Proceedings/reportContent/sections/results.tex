\section*{Results}
In this section, the outcomes generated by our Apache Spark-based streaming system will be presented. The primary results include EWMA drift detection, identification of Z-score anomalies, recognition of Bollinger Band breaches, and the provision of actionable trading suggestions. Our system, deployed on AWS, successfully processed and analyzed streaming TSLA data in real-time by utilizing Apache Spark Structured Streaming. The implementation included three distinct anomaly detection techniques—EWMA, Z-score, and Bollinger Bands—and visualized them through an interactive dashboard.

\subsection*{Streaming EWMA Trends}
\begin{figure}[H]
    \centering
    \includegraphics[width=0.95\linewidth]{images/EWMA.jpeg}
    \caption{The EWMA values for TSLA, computed over sliding windows. A consistent upward drift is observed through mid-May.}
    \label{fig:ewma}
\end{figure}

\vspace{1em}

\subsection*{Z-score Based Anomaly Detection}
\begin{figure}[H]
    \centering
    \includegraphics[width=0.95\linewidth]{images/Z_score.jpeg}
    \caption{Z-score calculations for stock prices over five minute windows. No Z-score breaches beyond ±2 detected, indicating price stability.}
    \label{fig:zscore}
\end{figure}

\vspace{1em}

\subsection*{Bollinger Band Evaluation}
\begin{figure}[H]
    \centering
    \includegraphics[width=0.95\linewidth]{images/Streaming_bollinger.jpeg}
    \caption{TSLA price vs. its Bollinger Bands. All values remained within bounds, suggesting low short-term volatility.}
    \label{fig:bbands}
\end{figure}

\vspace{1em}

\subsection*{Dashboard Visualization}

\begin{figure}[H]
    \centering
    \includegraphics[width=0.95\linewidth]{images/Dashboard1.jpeg}
    \caption{The interactive dashboard displaying the price stream, EWMA, Bollinger Bands, and detected anomalies.}
    \label{fig:dashboard1}
\end{figure}

\begin{figure}[H]
    \centering
    \includegraphics[width=0.95\linewidth]{images/Dashboard2.jpeg}
    \caption{A zoomed-in dashboard view showing a localized anomaly spike. The suggested action advises bearish strategy.}
    \label{fig:dashboard2}
\end{figure}

\vspace{1em}

\subsection*{Interpretation}
The EWMA score effectively tracked smoothed price trends using an alpha value of 0.3. On May 14, a noticeable spike in the actual stock price above the EWMA value indicated upward momentum. The EWMA curve adjusted gradually, providing clear trend confirmation by filtering out the noise of short-term price fluctuation.
\newline

Overall, the EWMA provided a reliable baseline for identifying sustained market sentiment. When stock prices remained persistently above or below the EWMA, it indicated bullish or bearish sentiment, respectively. Across the analyzed periods:
\begin{itemize}
    \item EWMA trends showed an upward drift between May 13–15, followed by stabilization.
    \item Z-scores generally fluctuated near 0, with no statistical anomalies ($|z| > 2$) observed.
    \item Stock prices consistently remained within the upper and lower Bollinger Bands.
    \item Dashboard alerts correlated with early price drops and subsequent returns to mean behavior.
\end{itemize}

The system performed reliably during periods of stable price movement. In live anomaly scenarios, alerts marked by red “X” symbols on the dashboard would recommend actionable trading strategies (buy, sell, or hold). Additionally, if the z-score exceeded the threshold value of 2, the system would alert users to a significant anomaly. However, throughout our testing period, most z-scores remained within ±1.5, indicating stable market conditions. Notably, a price jump on May 12 triggered a z-score alert, marking one of the few instances where the threshold was breached.
\newline

Our third anomaly detection approach utilized Bollinger Bands to identify rapid price deviations. For the duration of our testing phase, prices consistently stayed within the computed bands. Prices occasionally approached the upper band, however, hinting at elevated volatility during that time. The Bollinger Bands thus served well as volatility thresholds. Although no breaches occurred, instances where prices neared the bands often aligned with z-score alerts, demonstrating the convergence of these detection signals.
\newline

